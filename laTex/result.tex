\section{結果と考察}
\subsection{PPG1500を開始剤とする精密ポリ乳酸の合成}
PPG1500を開始剤とするポリ乳酸の分子量,分子量分布を表\ref{table:PLA-b-PPG1500synthesis}にまとめた。
$\textrm{[M]/[I]} < 200$の合成では,モノマー転換率が99$\%$以上かつ,PDIが1.19以下と定量的な
合成が可能であった。また,$\textrm{[M]/[I]} = 20$では,ポリマーが析出しなかった。
\subsection{PPG3000を開始剤とする精密ポリ乳酸の合成}
PPG3000を開始剤とするポリ乳酸の分子量,分子量分布を表\ref{table:PLA-b-PPG3000synthesis}にまとめた。
開始剤をPPG1500としたときの合成と比較して,モノマー転換率が低下した。
また,PPG1500と同様に,$\textrm{[M]/[I]} = 20$では,ポリマーが析出しなかった。
\begin{table}[h]
    \caption{PLA-b-PPG1500の合成結果}
    \label{table:PLA-b-PPG1500synthesis}
    \centering
      \begin{tabular}{cccccccc}
        \hline
        run &  [M]/[I] & Conversion & Yield & \textit{M}${}_\textrm{n}$(th) & \textit{M}${}_\textrm{n}$ & \textit{M}${}_\textrm{w}$ & PDI\\
        &   & /\% & /\% & /\si{\g} \si{mol^{-1}} & /\si{\g} \si{mol^{-1}} & /\si{\g} \si{mol^{-1}} & \\
        \hline\hline
        1 & 20 & 100 & 4383 & - & - & - & - \\
        2 & 50 & 99 & 65 & 8707 & 13780 & 15552 & 1.13 \\
        3 & 100 & 99 & 75 & 15913 & 19501 & 23178 & 1.19\\
        4 & 200 & 92 & 67& 30326 & 32200 & 40756 & 1.27\\
        5 & 500 & 84.4 & 54 & 73565 & 51258 & 57326 & 1.12\\
        \hline
      \end{tabular}
\end{table}

\begin{table}[t]
    \caption{PLA-b-PPG3000の合成結果}
    \label{table:PLA-b-PPG3000synthesis}
    \centering
    \begin{tabular}{cccccccc}
      \hline
      run &  [M]/[I] & Conversion & Yield & \textit{M}${}_\textrm{n}$(th) & \textit{M}${}_\textrm{n}$ & \textit{M}${}_\textrm{w}$ & PDI\\
      &   & /\% & /\% & /\si{\g} \si{mol^{-1}} & /\si{\g} \si{mol^{-1}} & /\si{\g} \si{mol^{-1}} & \\
      \hline\hline
      1 & 20 & 80 & - & - & - & - & - \\
      2 & 50 & 81 & 31 & 8268 & 13499 & 14341 & 1.14 \\
      3 & 100 & 89 & 54 & 15498 & 19240 & 22799 & 1.19\\
      4 & 200 & - & 37 & - & 24152 & 28996 & 1.20\\
      5 & 500 & 66 & 44 & 49546 & 28798 & 38546 & 1.34\\
      \hline
    \end{tabular}
\end{table}

