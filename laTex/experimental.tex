\section{実験}

\subsection{試薬}
モノマーにL-lactide(L-LA),開始剤にポリプロピレングリコール(PPG),
触媒にN,N-ジメチル-4-アミノピリジン(DMAP)および
そのトリフルオロスルホン酸塩(DMAPH \cdot HOTf)を用いた。
L-LAは,トルエンを溶媒として3回再結晶化して精製したものを使用した。
PPGは市販品をそのまま使用した。
DMAPは減圧下で加熱し,再結晶させたものを,
減圧乾燥後,窒素雰囲気下で保存して使用した。

\subsection{バルクでのラクチド開環重合}
L-LAの開環重合法を次に示す。
まず,二ツ口梨型フラスコを減圧下でフレームドライし,
室温まで冷却した後窒素雰囲気にした。そこに開始剤(PPG)を
任意の量,サンプル瓶に滴下し,ジクロロメタン(\ce{CH2Cl2})
で溶解したものを加えた後,フラスコが室温と同程度になるまで,溶液を
揮発させた。つづいて,DMAP(6.1\si{\mg},0.05\si{\mmol}),
DMAPH\cdot HOTf(13.6\si{\mg}, 0.05\si{\mmol}),
L-LA(720.7\si{\mg}, 5.0\si{\mmol})の順に加え,窒素気流下,
100\si{\degreeCelsius}オイルバス中で加熱撹拌した。
一時間後に$^1$H-NMR分析のために,少量のcrude反応物を採取した。
残りの反応物を3.5\si{\mL}の\ce{CH2Cl2}に溶解させたものを,
冷メタノール中に滴下し析出させ,白色固体状ポリマーを得た。
\subsection{生成物の分析}

\subsubsection{$^1$H-NMR分析}
JEOL製JML-ECA(600\si{\MHz}, CD\ce{Cl3}溶媒)により,
得られた生成物の$^1$H-NMR分析を測定した。反応率は,
モノマー中のメチルプロトンと,ポリマー中のメチルプロトンの
積分比から算出し,理論分子量は,の式から算出した。


\subsubsection{GPC分析}

