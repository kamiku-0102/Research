%jsarticleはchapterの定義なし
\section{緒言}
\subsection{プラスチックの現状}
軽くて丈夫かつ成形加工性に優れ安価なプラスチックは、
年間約4億t生産され、今後も生産量が増大していくと予想されている。
一方で金属やガラスといった他資源と異なり、多くの物が混ざっていることから
分別や再利用が困難であることから廃プラスチックの処理が課題となっている。
当初、廃プラスチックは埋め立てにより処理していたが、高度経済成長に伴う
廃棄量の急増による埋め立て処分場の不足から焼却処分が行われるようになった。
後に焼却時にダイオキシンが発生していることが分かったが、
2000年1月に「ダイオキシン類対策特別措置法」が施行されてからは、廃棄物処理施設
からのダイオキシン類排出量は年々減少し、現在は施行前の200分の1程度まで
低減されている。
国内では廃プラスチック削減のための取り組みが行われ、消費者が分別排出したものを、
市町村が分別収集し、それを事業者がリサイクルを行う「容器リサイクル法」の制定や
従来の3R(リデュース、リユース、リサイクル)に加え、
再生利用・バイオマスプラスチックに関する6つのマイルストーンを掲げた
「プラスチック資源循環戦略」の提唱がなされている。

\subsection{生分解性プラスチック}

\subsection{ポリ乳酸}
