\begin{abstract}
    ポリ乳酸(PLA)は植物由来かつ生分解性を持つことから、 既存プラスチックの代替材料として期待されている。
    ポリ乳酸は乳酸の二量体ラクチドをスズ触媒を用いて開環重合することで合成される。
    しかし、生成物にスズが残存することによる耐候性および耐熱性の低下や
    分子量が5万程度で分子量分布指数(\textit{M}${}_\textrm{w}$/\textit{M}${}_\textrm{n}$)が2程度と
    重合制御されていないことから、分子量と物性の関係が明確でないことが問題として挙げられる。
    そこで、本研究室では高活性、高効率かつ除去の容易な有機触媒を開発し、
    分子量、分子量分布および化学構造を制御したポリ乳酸の精密合成に成功した。
    既存プラスチックの代替に至らないもう一つの理由として硬くて脆く成形が困難であることがある。
    可塑剤の添加によるポリ乳酸の柔軟性改善が行われてきたが、
    引張強度が著しく低下することが分かっている。
    先行研究では柔軟なポリエチレングリコール(PEG)を開始剤とするPLA/PEGブロック共重合体を
    市販の高分子量のポリ乳酸に添加することである程度引張強度を保持したまま、
    伸び率が200\%を超えることが明らかとなった。
    そこで、PEGに比べて剛直なポリプロピレングリコール(PPG)を開始剤とするPLA/PPGブロック共重合体を
    市販のポリ乳酸の添加剤とすれば、引張強度がポリ乳酸と同程度のまま伸びる材料ができると考えた。
    ここでは、添加剤として用いるPLA/PPGブロック共重合体を研究室独自の酸塩基有機触媒を用いて精密合成を試みた。
\end{abstract}