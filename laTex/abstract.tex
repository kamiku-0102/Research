\begin{abstract}
    ポリ乳酸(PLA)は、その植物由来かつ生分解性の特性から、既存のプラスチックに代わる素材として注目されています。
    ポリ乳酸は、乳酸の二量体であるラクチドをスズ触媒を用いて開環重合することで合成されます。
    しかし、生成物にスズが残存することで耐候性や耐熱性が低下し、
    また、分子量が約5万程度であり、分子量分布指数(\textit{M}${}_\textrm{w}$/\textit{M}${}_\textrm{n}$)が約2という
    重合の制御が不十分であるため、分子量と物性の関係が不明確です。
    そこで、本研究では、高活性で効率的かつ容易に除去可能な有機触媒を開発し、ポリ乳酸の精密合成に成功しました。
    さらに、ポリ乳酸は硬くて脆く、成形が困難であるという課題も存在します。
    これに対処するため、可塑剤の添加が試みられてきましたが、その結果、引張強度が著しく低下することが知られています。
    先行研究では、柔軟なポリエチレングリコール(PEG)を開始剤とするPLA/PEGブロック共重合体を
    市販の高分子量のポリ乳酸に添加することで、一定の引張強度を保持しつつ、伸び率が200\%を超えることが示されました。
    そこで、PEGよりも剛直なポリプロピレングリコール(PPG)を開始剤とするPLA/PPGブロック共重合体を
    市販のポリ乳酸に添加することで、引張強度を維持しながらも材料の伸びを実現することを目指しました。
    本研究では、研究室独自の酸塩基有機触媒を用いて、この目標を達成するための
    PLA/PPGブロック共重合体の精密合成を試みました。
\end{abstract}